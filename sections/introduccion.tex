\documentclass{article}
\newenvironment{standalone}{\begin{preview}}{\end{preview}}
\usepackage{../includes}

\begin{document}
\begin{standalone}

  \section{Introducción}

  \subsection{Antecedentes}

  Aplicaciones basadas en posicionamiento satelital como Google Maps son muy útiles para obtener direcciones de navegación dentro de una ciudad, de un país, o incluso entre países.
  Pero a menor escala, para navegación dentro de edificios en donde puede haber numerosas habitaciones y pasillos, como un museo, un hospital, una planta industrial o un centro comercial, la tecnología GPS no podría aplicarse debido a la baja precisión, del orden del metro, y a la pobre recepción de señal dentro de edificios.

  Existen distintos métodos para solucionar el problema de la navegación en interiores.
  Entre ellos, se pueden mencionar técnicas de triangulación con la intensidad de señales de Wi-Fi recibidas, estimación con sensores de odometría como acelerómetros, magnetómetros y giróscopos, detección directa de etiquetas de ubicación por infrarrojo, ultrasonido, Bluetooth o sensores ópticos y, por último, técnicas de visión artificial.
  Muchas veces estos métodos se combinan para lograr mejores resultados.

  A nivel comercial, aún no existe alguna solución ampliamente utilizada para la navegación de interiores.
  Pero existen investigaciones y proyectos particulares que ofrecen diversas soluciones a este problema.
  Se puede mencionar a \citeauthor{slamNavARM} \cite{slamNavARM} y a \citeauthor{indoorNavRacoons} \cite{indoorNavRacoons} que han propuesto navegación usando métodos de SLAM con cámaras y sensores inerciales aprovechando la tecnología \textit{ARCore} de Google.
  Sus trabajos sirven de guía para el presente proyecto.

  \subsection{Propuesta}

  En este proyecto, se propone realizar una aplicación de realidad aumentada para teléfonos para la asistencia de navegación en interiores.

  En ella, el usuario selecciona a dónde quiere dirigirse dentro de un edificio y se le indicará las direcciones a seguir de tres formas.
  Por un lado, mediante flechas en un entorno de realidad mixta.
  Por otro lado, por medio de un minimapa virtual que le indicará la posición actual y el camino a seguir.
  Por último, se reproducen instrucciones de voz.

\end{standalone}
\end{document}
