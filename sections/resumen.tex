\documentclass{article}
\newenvironment{standalone}{\begin{preview}}{\end{preview}}
\usepackage{../includes}

\begin{document}
\begin{standalone}
  En aplicaciones robóticas colaborativas, especialmente carga y descarga de máquinas herramienta, los brazos robóticos deben manipular una amplia variedad de piezas cilíndricas.
  La mayor parte de las pinzas disponibles poseen dos dedos que pueden agarrar piezas de diversas formas, pero fallan en agarrar firmemente y centrar piezas cilíndricas.

  En el marco de unas pasantías en la empresa OnRobot A/S, se diseñó una pinza de tres dedos que centra automáticamente y sujeta con firmeza piezas cilíndricas de hasta 15kg.
  Esta pinza, cuya versión final fue sacada al mercado con el nombre 3FG15, está diseñada para ser el efector final de un robot serie colaborativo o pequeño.
  El UR10, un brazo de seis grados de libertad con articulaciones rotatorias que, con una capacidad de carga de hasta 10kg y un alcance de hasta 1300mm, es uno de los robots industriales colaborativos más extendidos en el mercado.

  En el presente trabajo, se propone modelar la cinemática y dinámica del robot UR10 y la pinza de tres dedos para una aplicación sencilla de carga y descarga, con especial énfasis en los dos últimos grados de libertad del brazo y el único grado de libertad de la pinza.
  También se plantea realizar el diseño mecánico, en particular la selección de actuadores y sensores, de la pinza y elegir un método de control para el brazo y la pinza.
  Finalmente, se busca realizar una simulación del robot y la pinza en la aplicación descripta.
\end{standalone}
\end{document}
