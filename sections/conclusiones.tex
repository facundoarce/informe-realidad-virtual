\documentclass{article}
\newenvironment{standalone}{\begin{preview}}{\end{preview}}
\usepackage{../includes}

\begin{document}
\begin{standalone}

\section{Conclusiones}

En el presente proyecto se desarrolló una aplicación para teléfonos Android en el entorno Unity en lenguaje C\texttt{\#} para la asistencia a la navegación en interiores.

El abordaje del desarrollo de la aplicación se dividió en varios subsistemas. En primer lugar, un sistema de seguimiento de posición provisto por la plataforma ARCore basada en algoritmos de localización y mapeo simultáneos SLAM a partir del procesamiento de imágenes de la cámara y de datos de sensores inerciales.
También se cuenta con un sistema de calibración por decodificación de códigos QR para la ubicación inicial y subsecuentes relocalizaciones para corregir errores de posicionamiento.
Se implementó un sistema de navegación para encontrar el camino hasta la ubicación.
El camino es indicado al usuario, por un lado, marcando la línea a seguir en el minimapa y, por otro lado, colocando objetos de realidad aumentada como flechas para indicar la dirección a seguir y un pin para indicar que se ha llegado a destino.
Se realizó también un sistema de comunicación cliente-servidor a través de Bluetooth para descargar datos de navegación desde otro dispositivo de manera remota.
El útimo subsistema es la interfaz gráfica que se diseñó para que el usuario pueda controlar y entender las funcionalidades de la aplicación de forma fácil e intuitiva.

Se probó la aplicación en el departamento del autor del proyecto y pudo comprobarse su correcto funcionamiento, así como las dificultades y errores intrínsecos a la solución elegida para la navegación de interiores.

Se deja como trabajo futuro la posibilidad de que el mapa completo se descargue desde un dispositivo externo y no solamente los puntos de referencia, la adición de indicaciones de voz para la navegación, la mejora del sistema de calibración calculando la distancia y orientación del usuario respecto al código QR para una reubicación más precisa, y sobre todo, la mejora de la precisión del sistema de seguimiento.
Las técnicas de SLAM con cámaras son todavía muy recientes y se encuentran en constante mejora.
Es un deseo del autor que, en un futuro cercano, esta tecnología esté lo suficiente madura para el uso comercial de asistentes de navegación en interiores.

\end{standalone}
\end{document}
